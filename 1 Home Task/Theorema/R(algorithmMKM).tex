\documentclass[12pt]{article}
\usepackage[pdftex]{graphicx}
\usepackage[utf8x]{inputenc} 
\usepackage{amsfonts}
\usepackage[left=1.5cm,right=1.5cm,top=1.5cm,bottom=1.5cm]{geometry}
\usepackage[russian]{babel}
\DeclareGraphicsExtensions{.pdf,.png,.jpg}
\title{Алгоритм МКМ}  
\date{10/10/2017}  
\author{Kailiak Eugene}

\begin{document}
\maketitle
\begin{enumerate}


\item Заметим, что рассматриваемые суммы в потенциалах - части пропускной способности разреза нашего графа. Это верно потому, что граф слоистый. Так как $c(e) - f(e) \geq 0$ для любого ребра $e$, то пропускная способность разреза между двумя слоями (не содержащими $s$ и $t$) $C(S, T) = \Sigma_{u \in V_i}\Sigma_{v \in V_{i + 1}} (c(u, v) - f(u - v))$ Поэтому если мы возьмём только одну вершину во внешней сумме, то она станет не больше. А так как максимальный поток в графе равен пропускной способности минимального разреза, то поток такой величины мы в графе пропустить можем. \\
Докажем теперь, что можно пропустить через вершину $r$. Так как по определению потенциала мы можем пропустить поток этой величины по инцидентным ребрам к вершине $r$, запустим такие потоки вдоль рёбер. Рассмотрим теперь вершины, в которые пришли новые потоки. Так как величина пущенного по ребру потока меньше величины потенциала $r$, а он минимальный во всём графе, то пришедший в новый слой поток меньше потенциала каждого ребра в этом слое. Аналогично и для всех слоёв (мы пускаем поток "в две стороны" от нашей вершины, но на суть это не влияет). То есть в каждом слое в одну вершину могло влиться дополнительно не больше $\phi(r)$ потока. А по определению потенциала через минимум пропускных способностей в остаточной сети в и из вершины, мы можем пустить этот поток дальше, пока не дойдём до $s$ и $t$. Следовательно, такой поток пустить можно
\item
Алгоритм будет заключаться в следующем: на каждой итерации по имеющейся у нас сети с посчитанными потенциалами и выброшенными вершинами с нулевыми потенциалами строим слоистую сеть. В ней будем искать блокирующий поток, находя вершину с наименьшим потенциалом и пуская через неё поток $\phi(r)$, попутно удаляя из слоистой сети вершины с нулевым потенциалом и инцидентные им рёбра, обновляя потенциалы при удалении и пропускании потока. На каждой итерации мы ходим по какому-то количеству рёбер, количество которых не больше, чем $|V|$ (мы считаем, что от кратных ребёр мы заранее избавились), при этом $|E_i|$ из них мы удаляем. Тогда за один поиск блокирующего потока мы сделаем операций $\Sigma_i {O(|V|) + |E_i|} = O(|V|^2)$ Всего итераций $O(|V|)$, поэтому общая асимптотика будет $O({V}^3)$

\end{enumerate}

\end{document}