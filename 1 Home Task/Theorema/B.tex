\documentclass[12pt]{article}
\usepackage[pdftex]{graphicx}
\usepackage[utf8x]{inputenc} 
\usepackage{amsfonts}
\usepackage[left=1.5cm,right=1.5cm,top=1.5cm,bottom=1.5cm]{geometry}
\usepackage[russian]{babel}
\DeclareGraphicsExtensions{.pdf,.png,.jpg}
\title{B}  
\date{18/09/2017}  
\author{Kailiak Eugene}

\begin{document}
\maketitle
Сделаем из нашего двудольного графа сеть (добавим сток и исток), ребра ориентируем из левой доли в правую и назначим им пропускную способность 1. От стока проведём аналогичные рёбра в левую долю, а из правой доли проведём рёбра к стоку. В этой сети можно использовать теорему Форда-Фалкерсона и сказать, что величина максимального потока равна пропускной способности минимального разреза. Также будем использовать Лемму о том, что поток не превосходит пропускную способность минимального разреза. \\
Пусть $n = |L|$. Докажем в одну сторону. Рассмотрим разрез $(S,T)$ в котором $m$ вершин из левой доли и $k$ вершин из правой. Разберём случаи и посчитаем пропускную способность.: 
	\begin{enumerate}
\item $m \leq l$. Тогда пропускная способность содержит $n-m$ ребёр из $s$ в первую долю, и $l$ рёбер из вершин второй доли, лежащих в $S$, ведущих в $t$. Тогда пропускная способность не меньше чем $n - m + l = n + (l - m) \geq n$, т.к. $l - m \geq 0$
\item $m > l$. Тогда по условию теоремы $N(A) \leq |A|$, что означает, что эти $m$ вершин первой доли соединены с не меньшим числом вершин правой доли, то есть существует как минимум $m - l$ рёбер из $S$ в $T$ + $(n-m)$ рёбер в левую долю из $s$ и $l$ рёбер из правой доли в $t$. Тогда пропускная способность $\geq (n-m) + l + (m-l) = n$ 
\end{enumerate}
Итак, в любом случае пропускная способность разреза не больше $n$, поэтому в одну сторону доказали.
Докажем во вторую. Пусть существует паросочетание размера n. Значит, пропускная способность минимального разреза $\geq n$. Пусть дано множество $A$ (докажем для любого). Возьмём разбиение $(S,T)$, где $S = A \cup N(A)$. Тогда аналогично посчитаем минимальную пропускную способность: $(n - |A|)  + N(A) \geq n \Rightarrow N(A) - |A| \geq 0 \Rightarrow N(A) \geq |A| $, что и требовалось доказать.
	

\end{document}