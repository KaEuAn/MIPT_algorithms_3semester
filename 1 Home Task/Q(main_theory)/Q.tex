\documentclass[12pt]{article}
\usepackage[pdftex]{graphicx}
\usepackage[utf8x]{inputenc} 
\usepackage{amsfonts}
\usepackage[left=1.5cm,right=1.5cm,top=1.5cm,bottom=1.5cm]{geometry}
\usepackage[russian]{babel}
\DeclareGraphicsExtensions{.pdf,.png,.jpg}
\title{Кубическое проталкивание предпотока}  
\date{12/10/2017}  
\author{Kailiak Eugene}

\begin{document}
\maketitle
\begin{enumerate}
\item
Пусть $\Phi = max\{h(v) | v \in V\{s,t\}, v - $избыточная вершина$\}$. Если высоты за один цикл не поменялись, то тогда все избытки, которые были в вершинах, сливались вниз, поэтому потенциал уменьшился. Если потенциал остался таким же, то есть как минимум одна вершина, которая поднялась, потому что иначе избыточный поток в верхней вершине просто бы перешёл вниз, сама вершина перестала быть избыточной (свелось бы к первому случаю). Если потенциал увеличился, то есть как минимум одна вершина, у которой как минимум точно так же увеличилась высота. Сверху потенциал ограницен $O(|V|)$, как уже было доказано. Поэтому количество итераций, когда $\Phi$ не уменьшается, не больше $O(|V|^2)$. Аналогично с количеством итераций, где потенциал уменьшается. Следовательно, и всего меньше $O(|V|^2)$ \\
discharge делает максимум одно ненасыщающее проталкивание (потому что тогда мы сразу же выйдем из discharge после него), а для остальных операций мы доказали, что их количество не превышает куб. Следовательно, такой алгоритм работает за $O(V^3)$
\item
Пусть у нас в очереди множество вершин $U, M = V \backslash U$. Тогда можем разбить этот алгоритм и рассматривать его как вариант предыдущего, в котором сразу же пропускаются вершины, в которых нет избытка (из них не надо делать discharge). Тогда к тому моменту, когда из очереди мы достанем все вершины множества $U$ и сделаем для них discharge, это будет эквивалентно ровно одному циклу for из первого варианта. При этом так как мы сразу же добавляли получаемые новые переполненные вершины, мы подготовили очередь, которая соответствует второй итерации Simple For. Поэтому асимптотика будет такая же - $O({V}^3)$
\end{enumerate}
\end{document}